% Complex systems simulations are becoming increasingly common in formal and informal STEM learning environments~\cite{smordal2012hybrid}. These simulations present scientific phenomena in a manner that bridges principles of science and the firsthand experience of emergent, real-world outcomes. However, the open-ended and exploratory nature of these simulations present challenges to teachers' understanding of students' learning. Students' actions  have  immediate and long-term effects on the simulation leading to a rich array of emergent outcomes. Teachers may wish to discuss students' interactions to highlight salient learning opportunities, but if there are too many `moving parts' to the simulation, this becomes a challenging ideal.
%
% This paper presents an automatic method for extracting salient periods from the log files that are generated by complex exploratory learning environments (ELE). Our goal is to present relevant summaries of the system dynamics such that teachers can effectively engage students in discussions that stem from their own experiences with the simulations. We study an application of Switching State Space Models (SSSM) to the task of extracting salient periods from a mixed reality ELE, Connected Worlds, installed at the New York Hall of Science. SSSMs~\cite{ghahramani2000variational} are a class of models for time-series data where the parameters controlling a linear dynamic system switch according to a discrete latent process. These models have seen use in a wide variety of domains including control~\cite{ikoma2002tracking}, statistics~\cite{cappe2009inference}, econometrics~\cite{giordani2007unified} and signal processing~\cite{kim1999state}. SSSMs combine hidden Markov and state space models to capture \textit{regime} switching in non-linear time series~\cite{whiteley2010efficient}. The intuition is that a system evolves over time but may undergo a regime change that results in an intrinsic shift in the system's characteristics. Allowing for discrete points in time where the dynamics change, enhances the power of the simple linear models to capture more complicated dynamics. We propose that regime switching models also help to increase the \textit{interpretability} of large and complex systems by automatically segmenting a time series into regions of approximately uniform dynamics. The result is that a complex session is broken into smaller periods that are more readily understood upon reflection on the session.
%
% In this paper we introduce the Connected Worlds ELE and explain why teachers might need assistance when leading a discussion with the students where they reflect upon their actions. We expound on the SSSM and propose a method for decomposing a complex time series into smaller periods aiming to assist teachers when reflecting on a session with a class. We lastly present results showing the efficacy of our approach on synthetic data, as well as a preliminary study showing that the model output is human interpretable.
