Complex systems simulations are becoming increasingly common in formal and informal STEM learning environments~\citep{smordal2012hybrid}. These simulations present scientific phenomena in a manner that connects pedagogical learning outcomes to the firsthand experience of real-world outcomes. However, the open-ended and exploratory nature of these simulations present challenges to teachers' understanding of students' learning. Students' actions have both immediate and long-term effects on a simulation, leading to a rich array of potential outcomes. Teachers may wish to lead discussions on students' interactions to highlight salient learning opportunities, but if there are too many `moving parts' to the simulation, this becomes a challenging ideal.

This thesis presents an investigation into and a case study for using artificial intelligence techniques to extract useful information from the log files of exploratory learning environments. We study an exploratory learning environment that simulates an ecosystem and aims to teach students about systems, systems thinking and how their actions can have possible long term and spatially removed effects on the system. The system states that result from students' interactions with the simulation are logged and form a time series of data that represent the dynamical responses of the complex system to the group specific interactions. Our hypothesis is: \textit{complex time series log data can be decomposed into periods that individually are readily understood}. Together these periods explain the dynamics that ensued, but individually they might explain salient fragments or examples of the larger causal cycles at play.

We therefore study methods for extracting salient periods from the log files that are generated by complex exploratory learning environments. Switching state space models~\citep{ghahramani2000variational} are a class of models for time series data where the parameters controlling a linear dynamic system switch according to a discrete latent process. These models have been used in a wide variety of domains including control~\citep{ikoma2002tracking}, statistics~\citep{cappe2009inference}, econometrics~\citep{giordani2007unified} and signal processing~\citep{kim1999state}. They combine hidden Markov and state space models to capture \textit{regime} switching in non-linear time series data~\citep{whiteley2010efficient}. The intuition is that a system evolves over time but may undergo a regime change that results in an intrinsic shift in the system's characteristics. Allowing for discrete points in time where the dynamics change enables simple linear models to capture more complicated and non-linear dynamics. We propose that regime switching models also help to increase the \textit{interpretability} of large and complex systems by automatically segmenting a time series into regions of approximately uniform dynamics. The result is that a complex session is broken into smaller periods that are more readily understood when one reflects on the session.

The switch based model formalizes a procedure for decomposing a time series into coherent periods broken by the change points between those periods. However, inference in these models is extremely challenging. The tightly coupled nature of the regime parameters and the locations of switch points presents a computationally intractable problem. We present a novel inference algorithm for learning the parameters that are associated with the unknown individual regimes. Given the regime specific parameters, inference over the change points can be conducted using standard techniques from time series modeling.

Our eventual goal is to present relevant summaries of the system dynamics such that teachers can effectively engage students in discussions that stem from their own experiences with the simulation. To this end, we investigate the human interpretability of the resulting periods that are identified by the presented inference algorithm. We use the regime specific parameters to generate a short description of the dynamics that were present in a given period. We conduct two validation experiments to evaluate the extent to which the inferred switch points and the associated regime descriptions are comprehensible to external human validators.

The thesis proceeds as follows. In Chapter~\ref{ch:1}, we introduce Connected Worlds, the exploratory learning environment that is presented as a case study for this investigation. Chapter~\ref{ch:2} presents a detailed description of the switching state space model. This chapter elaborates on hidden Markov and state space models and presents the switch models that combine these two baselines. In addition, Chapter~\ref{ch:2} presents the past work for performing maximum likelihood estimation of the parameters that are associated with the switching state space models. Lastly, this chapter describes the inference algorithm that we use for extracting salient periods from the Connected Worlds log data. Chapter~\ref{ch:3} describes the evaluation of the inference algorithm. We generate synthetic data and test the ability to correctly detect change points in a given time series. We then design two experiments that aim to test the coherence and interpretability of the model's output. Finally, Chapter~\ref{ch:4} presents two avenues for future research. Work in Bayesian non-parametrics presents an exciting possibility for applications in these domains. Moreover, future work will need to design and evaluate tools that incorporate such information and are appropriate for classroom implementation.



